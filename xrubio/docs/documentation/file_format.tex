\documentclass[a4paper,10pt]{article}
\usepackage[utf8]{inputenc}
\usepackage{url}

% Title Page
\title{Pandora file format}
\author{Xavier Rubio-Campillo}


\begin{document}
\maketitle

\begin{abstract}
This document defines the format that Pandora uses to serialize a simulation.
\end{abstract}

\section{Introduction}

Pandora uses HDF5\footnote{\url{http://www.hdfgroup.org/HDF5/}} to store the contents of a simulation, both in distributed environments (thus avoiding bottlenecks) and serial executions. This is an structured binary format designed for scientific applications.

Each simulation developed with the library generated at least a \textbf{.h5} file, and one \textbf{.abm} file for every computer node used in the execution (they are numbered following the id of the node, i.e. \textit{agents-0.abm}, \textit{agents-1.abm}). The number of the \textbf{.h5} file and the directory where all the data is stored must be specified at execution time, usually inside the XML configuration.

\section{Base \textbf{.h5} file}

This file provides information about global parameters of the simulation, and the raster maps.

\subsection{Global Parameters}

This dataset comprises 4 different values:


\begin{enumerate}
  \item numSteps (integer)
Total number of executed time steps.
  \item numTasks (integer)
Number of tasks used during the execution
  \item serializerResolution (integer)
How many time steps are executed before storing one.
  \item{size}
Total size of the environment
\end{enumerate}

As an example:
\begin{verbatim}
GROUP "/" {
   DATASET "global" {
      DATATYPE  H5T_STD_I32LE
      DATASPACE  SIMPLE { ( 1 ) / ( 1 ) }
      DATA {
      (0): 0
      }
      ATTRIBUTE "numSteps" {
         DATATYPE  H5T_STD_I32LE
         DATASPACE  SIMPLE { ( 1 ) / ( 1 ) }
         DATA {
         (0): 18000
         }
      }
      ATTRIBUTE "numTasks" {
         DATATYPE  H5T_STD_I32LE
         DATASPACE  SIMPLE { ( 1 ) / ( 1 ) }
         DATA {
         (0): 4
         }
      }
      ATTRIBUTE "serializerResolution" {
         DATATYPE  H5T_STD_I32LE
         DATASPACE  SIMPLE { ( 1 ) / ( 1 ) }
         DATA {
         (0): 360
         }
      }
      ATTRIBUTE "size" {
         DATATYPE  H5T_STD_I32LE
         DATASPACE  SIMPLE { ( 1 ) / ( 1 ) }
         DATA {
         (0): 1600
         }
      }
   }
\end{verbatim}

\subsection{Rasters}
The dataset contains all raster map information of the simulation. There are two types of rasters: Static Rasters and Dynamic Rasters. In the root element there exist a Dataset defining which of the Rasters that will be loaded are Static. It comprises a list of Attributes, whose identifiers are the Static Rasters (so, the ones that are not listed here are Dynamics):

\begin{verbatim}
    DATASET "staticRasters" {
      DATATYPE  H5T_STD_I32LE
      DATASPACE  SIMPLE { ( 1 ) / ( 1 ) }
      DATA {
      (0): 0
      }
      ATTRIBUTE "dem" {
         DATATYPE  H5T_STD_I32LE
         DATASPACE  SIMPLE { ( 1 ) / ( 1 ) }
         DATA {
         (0): 0
         }
      }
   }
\end{verbatim}


\subsubsection{Static Rasters}

Static Rasters are used by a Pandora simulation, but cannot be modified (i.e. GIS data like Digital Elevation Models, rivers and lakes, etc.).
They form a Group (whose identifier is the name of the Static Raster) inside the root element \textit{/}. This Group has an integer DataSet called 'values' with the size specified by the global parameter \textit{size}. Inside the dataset the information is provided as a bidimensional matrix, like:
\begin{verbatim}
 
GROUP "/" {
   GROUP "dem" {
      DATASET "values" {
         DATATYPE  H5T_STD_I32LE
         DATASPACE  SIMPLE { ( 264, 264 ) / ( 264, 264 ) }
         DATA {
         (0,0): -32768, -32768, -32768, -32768, -32768, -32768, -32768,
         (0,7): -32768, -32768, -32768, -32768, -32768, -32768, -32768,
         (0,14): -32768, -32768, -32768, -32768, -32768, -32768, -32768,
         (0,21): -32768, -32768, -32768, -32768, -32768, -32768, -32768,
         (0,28): -32768, -32768, -32768, -32768, -32768, -32768, -32768,
         (0,35): -32768, -32768, -32768, -32768, -32768, -32768, -32768,
         (0,42): -32768, 1575, 1262, 1526, 1175, 1080, 828, 536, 574, 913,
	  ..................
	  ..................
	  ..................
	  ..................
	  ..................
	  ..................
         (263,252): -32768, -32768, -32768, -32768, -32768, -32768, -32768,
         (263,259): -32768, -32768, -32768, -32768, -32768
         }
      }
   }
\end{verbatim}

\subsubsection{Dynamic Rasters}

Dynamic Rasters are exactly like the static ones, that can be modified by the agents or the world every time step. They are defined as a Group inside the root element whose identifier is the name of the raster. Inside this Group there is a dataset for every time step from 0 to \textit{numSteps}, with identifiers following this value (i.e. \textit{step0}, \textit{step1}, \textit{step2}, ...). Every time step dataset contains a bidimensional matrix of the global size:

\begin{verbatim}
    GROUP "resources" {
      DATASET "step0" {
         DATATYPE  H5T_STD_I32LE
         DATASPACE  SIMPLE { ( 264, 264 ) / ( 264, 264 ) }
         DATA {
         (0,0): 5, 1, 4, 1, 1, 3, 3, 2, 2, 5, 5, 3, 3, 4, 1, 0, 4, 4, 3, 4,
         (0,20): 1, 4, 5, 4, 3, 3, 2, 4, 2, 3, 3, 3, 3, 3, 1, 2, 4, 4, 3, 0,
         (0,40): 4, 4, 1, 2, 5, 0, 3, 4, 1, 0, 2, 3, 0, 4, 0, 5, 4, 1, 0, 1,
         (0,60): 3, 1, 3, 4, 3, 2, 4, 4, 1, 5, 4, 3, 4, 2, 1, 3, 3, 3, 0, 0,
         (0,80): 0, 1, 2, 3, 2, 1, 3, 3, 2, 3, 2, 2, 0, 2, 2, 2, 4, 1, 3, 3,
         (0,100): 1, 5, 4, 2, 4, 2, 4, 1, 1, 2, 2, 4, 4, 4, 1, 3, 0, 2, 3, 4,
	  ..................
	  ..................
	  ..................
	  ..................
	  ..................
	  ..................
         (263,229): 1, 4, 5, 0, 2, 2, 3, 4, 2, 4, 1, 3, 3, 5, 3, 4, 2, 2, 5,
         (263,248): 5, 2, 3, 3, 4, 0, 2, 2, 4, 2, 3, 2, 5, 1, 5, 4
         }
      }
      DATASET "step1" {
	  ..................
\end{verbatim}

\section{Agents (.abm) files}

Each task stores the agents that are inside their boundaries at a given time step (this is the reason why there is one file per task). Pandora generates a Group for each Agent type. Inside it there is a nested list of Groups, each one containing the information of a time step from 0 to \textit{numSteps}, with identifiers following this value (i.e. \textit{step0}, \textit{step1}, \textit{step2}, ...).
Finally, the time step group has a Dataset for every attribute of the agent that was stored. At least the id (string), x and y positions (integer) are stored, plus any additional attributes defined in the method \textit{registerAttributes}. Each attribute Dataset is a unidimensional vector with size equal to the number of agents for this particular time step and task (so all attribute Datasets inside a time step group have the same size), like:

\begin{verbatim}
GROUP "/" {
   GROUP "RandomAgent" {
      GROUP "step0" {
         DATASET "id" {
            DATATYPE  H5T_STRING {
               STRSIZE H5T_VARIABLE;
               STRPAD H5T_STR_NULLTERM;
               CSET H5T_CSET_ASCII;
               CTYPE H5T_C_S1;
            }
            DATASPACE  SIMPLE { ( 2 ) / ( H5S_UNLIMITED ) }
            DATA {
            (0): "RandomAgent_0", "RandomAgent_1"
            }
         }
         DATASET "resources" {
            DATATYPE  H5T_STD_I32LE
            DATASPACE  SIMPLE { ( 2 ) / ( H5S_UNLIMITED ) }
            DATA {
            (0): 5, 5
            }
         }
         DATASET "x" {
            DATATYPE  H5T_STD_I32LE
            DATASPACE  SIMPLE { ( 2 ) / ( H5S_UNLIMITED ) }
            DATA {
            (0): 32, 16
            }
         }
         DATASET "y" {
            DATATYPE  H5T_STD_I32LE
            DATASPACE  SIMPLE { ( 2 ) / ( H5S_UNLIMITED ) }
            DATA {
            (0): 191, 116
            }
         }
      } 
      GROUP "step1" {
	..................
\end{verbatim}

\section{Final considerations}

If you want to see some code capable of loading this data, take a look to the \textbf{SimulationRecord} class. This is the basic class to load a Pandora execution, and uses the HDF5 C library to read through the data of any simulated model. It is used by Cassandra (GUI application) as well as the set of tools inside directory \textit{Analysis}

\end{document}
